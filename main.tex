\documentclass[oneside, 12pt]{memoir}
\usepackage{verbatim}
\usepackage[labelsep=period, font=singlespacing, skip=12pt, format=hang, justification=RaggedRight, singlelinecheck=false]{caption}
\usepackage[T1]{fontenc}
\usepackage{times}
\usepackage{indentfirst}
\usepackage{thesis}
\usepackage{frontmattercontents}
\usepackage{listings}
\usepackage{graphicx}
\usepackage{hyperref}
\usepackage{layouts}

\graphicspath{{./images/}}

\lstset{ %
xleftmargin=.2in,
xrightmargin=.1in,
language=Python,                	% choose the language of the code
basicstyle=\footnotesize,       % the size of the fonts that are used for the code
numbers=left,                   % where to put the line-numbers
numberstyle=\footnotesize,      % the size of the fonts that are used for the line-numbers
stepnumber=2,                   % the step between two line-numbers. If it's 1 each line 
                                % will be numbered                                                         
numbersep=5pt,                  % how far the line-numbers are from the code
showspaces=false,               % show spaces adding particular underscores
showstringspaces=false,         % underline spaces within strings
showtabs=false,                 % show tabs within strings adding particular underscores
frame=single,	                % adds a frame around the code
tabsize=2,	                	% sets default tabsize to 2 spaces
captionpos=b,                   % sets the caption-position to bottom
breaklines=true,                % sets automatic line breaking
breakatwhitespace=false,        % sets if automatic breaks should only happen at whitespace
title=\lstname,                 % show the filename of files included with \lstinputlisting;
                                % also try caption instead of title
escapeinside={\%*}{*)},         % if you want to add a comment within your code
morekeywords={*,...}            % if you want to add more keywords to the set
}


\checkandfixthelayout
\begin{document} 
	\setdoctype{Polemic}
	\masters 
	
	
\settitle{Modeling Mobile Web Characteristics for Energy Optimized Delivery}
\setauthor{Troy Johnson}

\setdefdate{September 2014}
\setgraddate{December, 2015}

\setchair{Dr. Patrick Seeling, Ph.D.}
\setmembers{Dr. Patrick Kinnicutt, Ph.D. \\ Dr. Michael Stinson, Ph.D.}


\newcommand{\titlepage}{{
\clearpage
\thispagestyle{empty}

\begin{SingleSpace}
\centering
\cmutitle \\
\vspace{1.42in} 
\cmuauthor \\
\vspace{1.42in}
A thesis submitted in partial fulfillment of\\
the requirements for the degree of\\
Master of Science \\
\vspace{2.333in}
Department of Computer Science \\
\vspace{2in}
Central Michigan University\\
Mount Pleasant, Michigan \\
September 2014\\
\end{SingleSpace}
\clearpage}}


\newcommand{\signaturepage}{{
\clearpage
\thispagestyle{plain}
{\centering
Accepted by the Faculty of the College of Graduate Studies,\\
Central Michigan University, in partial fulfillment of\\
the requirements for the master's degree\\}
\vspace{12pt}
{\noindent
Thesis Committee:\\}
\setlength{\tabcolsep}{0mm}
\begin{tabular}{{@{}l @{\hspace{10mm}}l}}
\rule{92mm}{.3pt}  &  Dr. Patrick Seeling \\
\rule{92mm}{.3pt}  &  Dr. Patrick Kinnicutt \\
\rule{92mm}{.3pt}  &  Dr. Michael Stinson \\
Date: \rule{81mm}{.3pt} & \\
\rule{92mm}{.3pt}  &  Dean \\ 
Date: \rule{81mm}{.3pt} & \parbox[tl]{2in}{\vspace*{-32pt} College of Graduate Studies} \\
\end{tabular}\\

{\noindent
Committee:\\
 
}
\clearpage}}

\newcommand{\acknowledgementspage}{{%
\clearpage
\thispagestyle{plain}

{\centering
ACKNOWLEDGMENTS \\}
This work was sponsored by an Early Career Grant from the Office of Research and Sponsored Programs at Central Michigan University. I wish thanks to A. Sampson and R. Kohvakka for assistance in the development of the testbed measurement environment.
%\end{Spacing}
\clearpage}}


\newcommand{\abstractpage}{{
\clearpage
\thispagestyle{plain}

{\centering
ABSTRACT \\
\begin{SingleSpace}
\cmutitle \end{SingleSpace}
by Troy Johnson\\}

As mobile traffic and data consumption continue to rise, there is a growing need to investigate increased energy efficiency and optimizations to reduce the bandwidth when browsing the mobile web. Use cisco figure citations here???? To determine the reduction in energy consumption of mobile devices, there is also a need for a way to measure the energy consumption of mobile devices. By investigating the composition and characteristics of mobile web pages, statistical models can be derived for describing the characteristics of a typical mobile web page, such as the individual response sizes and expiration ages of responses that mobile browsers request for web pages.HTTP Archive will be a great source of data that may be utilized to derive models for describing the mobile web. Additionally, this pool of data is updated on a bimonthly basis, providing a constantly updated pool of data to update the developed models with and validate them. These models can then in turn be used to provide more accurate results when estimating the possible energy and bandwidth savings by using these models for generating artificial web pages that will contain characteristics that closely resemble those characteristics often found on the actual mobile web.  Investigating the models and data further, they can be extended to create prediction models that will describe the growing mobile web for future years. With these models in place, they can be applied to projects for optimizing energy and bandwidth consumption on mobile devices, such as the possible energy and bandwidth savings that can result from cache forwarding between desktop computers and mobile devices. To measure the possible energy consumption savings from these projects, a low cost test bed for measuring the power consumption of mobile devices can be employed as a baseline
\clearpage}}

	

\frontmatter
\setulmarginsandblock{0.75in}{1.25in}{*}
\titlepage
\clearpage
\pagebreak
\begin{Spacing}{2}
\acknowledgementspage
\clearpage
\pagebreak
\abstractpage
\clearpage
\pagebreak
\tableofcontents*
\clearpage
\pagebreak
\listoffigures
\clearpage
\pagebreak
\pagestyle{plain}
\checkandfixthelayout
\mainmatter
\counterwithout{figure}{chapter}
\addtocontents{toc}{\noindent CHAPTER \par}
\setulmarginsandblock{0.75in}{1.25in}{*} 



\chapter{INTRODUCTION}
\label{ch:introduction}

\section*{HTTP Archive}
\addcontentsline{toc}{section}{HTTP Archive}
HTTP Archive is an online repository of web performance information containing information on both desktop and mobile versions of websites. Information gathered includes all the details about the responses each webpage makes such as the response sizes, expiration age, HTTP Archive gathers their data using a private instance of WebPageTest (http://www.webpagetest.org).
%Notes: Add more description in demonstration section
%Pull more specifics from the powepoint, also more
%description on the proxy overhead paper
\chapter{An Inexpensive Testbed for Mobile Device Power Measurement} 
\label{ch:testbed}


\section*{Introduction}
%addcontentsline add section to table of contents
\addcontentsline{toc}{section}{Introduction}

Over the last few decades, there has been an enormous increase in the ubiquity of mobile devices. With this increase, has also come the increase in demand for data-driven services and Cisco, Inc. predicts this demand to continue \cite{VNI14} into the foreseeable future. The battery consumption of mobile devices represents a limiting bottleneck and thus  power optimizations suggestions have been suggested \cite{Qian:2012:PTM:2187836.2187844} to reduce energy consumption on mobile devices. Software-based energy profilers do exist \cite{DAmato:2011:EAE:2419622.2419929}, however they are not always feasible for implementing in a straightforward manner or desirable due to rapid development cycles. To overcome these barriers, a real world test bed can be implemented  to perform measurements for the power consumptions on mobile devices to obtain a real life gauge for power estimations of optimizations. In this chapter, I will discuss the development and set up of a testbed for this purpose.

\addcontentsline{toc}{section}{Hardware Configuration}
\section*{Hardware Configuration}

The main component in this testbed is the mobile device.
This can be realized by using a smartphone and replacing the
battery with connectors to a power supply; alternatively one
of the common development board packages, such as
Pandaboard (see www.pandaboard.org) or Wandboard (see
www.wandboard.org) packages. Development boards and smartphones can be utilized together with the
Android operating system, which provides log output via USB
to the measurement control device, which can be a regular PC
or another development board with Android debugging
support. The mobile device is then networked with a wireless access point, which allows for wired and wireless evaluations.
The switchable power supply has an external serial or USB
port to communicate the current and power in small time
intervals to the control device. A BK Precision
1696 switchable power supply is utilized, as it offers fine granularity in power, current, and time intervals. While other equipment, such as an Arduino with custom circuits, were used in other measurement approaches, these power supplies are common lab equipment and offer overall robust features.

\addcontentsline{toc}{section}{Software Configuration}
\section*{Software Configuration}
The software components are comprised of several Python scripts that execute the Android Debugging Bridge (ADB) and capture the output either to a local file or allow sending of the output to a remote receiver, as illustrated in  Figure \ref{fig:testbed_setup}. These scripts allow for easy customization on the locally connected control device or at a remote location, e.g., filtering by specific events in the log. Similarly, a locally executing script captures the output from the power supply and is enabled to forward the data to a remote location as well.

\begin{figure}
\centering
\includegraphics[scale=1,keepaspectratio,width=\columnwidth]{testbed_setup}
\caption{Illustration of measurement testbed.}
\label{fig:testbed_setup}
\end{figure}

\addcontentsline{toc}{section}{Demonstration Description}
\section*{Demonstration Description}
To demonstrate the usefulness of this testbed, two different aspects of the measurement setup were utilized using an example Android application that performs web requests (wired and wirelessly). This application makes requests to a local proxy server, either serially or in parallel, and the proxy server goes out and fetches what the phone requested.  The workflow for the application can be seen in figure \ref{fig:testapp_workflow}. Both, wired and wireless access scenarios were exhibited for accessing a remote web service and retrieving results in order to demonstrate the functionality of the testbed. With these demonstrations, real time visualizations of the data about power consumption were created and stored on log files on a remote computer where they can be readily parsed for automatic evaluation of application power consumption on the mobile device.

\begin{figure}
\centering
\includegraphics[scale=1,keepaspectratio,width=\columnwidth]{testapp_workflow}
\caption{Workflow of mobile application tested on testbed.}
\label{fig:testapp_workflow}
\end{figure}

%Notes:Talk about what exactly a proxy server is
\chapter{Power Consumption Overhead for Proxy Services on Mobile Device Platforms}
\addcontentsline{toc}{section}{Introduction}
\section*{Introduction}
Current predictions by Cisco show that the amount of data that users consume has increased significantly and will continue to increase into the foreseeable future \cite{VNI14}. Previous studies show that the network interfaces of mobile devices consume much of the limited battery life \cite{Carroll:2010:APC:1855840.1855861}. Thus, heavy research efforts have been poured into studying the possibilities of energy efficient mobile data delivery. Some research avenues have middleware that acts as an on-device proxy service to realize benefits or enable new interaction paradigms, such as display networks \cite{6174992} or mobile content sharing \cite{Seeling:2014:OES:2671189.2671194},\cite{6692468}. To determine whether or not local proxy servers result in a large overhead in terms of power consumption and time delays, the measurement framework testbed described in Chapter \ref{ch:testbed} can be implemented to determine what kind of overheads can be expected from local proxy servers.

\addcontentsline{toc}{section}{Methodology and Metrics}
\section*{Methodology and Metrics}

\addcontentsline{toc}{subsection}{Measurement Setup}
\subsection*{Measurement Setup}
At the core of the setup, a Pandaboard ES mobile
software development platform is utilized, which features a Texas Instruments OMAP 4460 dual core ARM Cortex-A9 processor with
1 GB of DDR2 RAM, SMSC 10/100 Mbps Ethernet port, and
LS Research WLAN/Bluetooth wireless module, next to other
components. (Please refer to http://www.pandaboard.org for
more details. The open-source Android distribution
(version 4.1.2, ‘Jelly Bean’)  is used as the operating system software for the mobile device.
The overall measurement setup in can be seen in Figure \ref{fig:proxy_setup}. The
Pandaboard is powered by a BK Precision 1696 switchable
power supply, which features serial port access to read out
voltage and ampere values over time. The power
supply is connected to a Linux desktop computer serial port which timestamps
the values obtained over time to measure the power consumption
incurred by the Pandaboard. The Pandaboard is connected
through a 1 Gbps maximum speed Ethernet campus network,
which eliminates potential bottlenecks. For wireless measurements,
an externally connected WLAN antenna is utilized to
connect to the campus network through a dedicated WLAN access point, again eliminating bottlenecks for the
amounts of data considered throughout. It's also important to note
that a combination of input devices and an external monitor
were connected as well.
On the server side, a locally hosted virtual
machine next to Internet-routed web requests is utilized. The local server employs the Debian Linux distribution as its' operating system
with the Apache2 HTTP server and the popular Video Lan
Client (VLC) as media streaming application.A preencoded
video sequence of the popular open-source movie Tears
of Steel (see http://www.tearsofsteel.org for more information) is streamed utilizing HTML5 video streaming. The video-only sequence
was transcoded offline into a resolution of 864 × 480 at 24
frames per second in the Theora video codec and encapsulated
using the OGG container format, both commonly utilized for
HTTP video streaming “on the web” and suitable for mobile
playout. The resulting video bit stream has a duration of 12
minutes, 14 seconds and an average bit rate of 1.42 Mbps.
The bit rate in turn falls well within range of the network
bandwidth capacity.


\addcontentsline{toc}{subsection}{Mobile SOCKSv5 Proxy Server}
\subsection*{Mobile SOCKSv5 Proxy Server}
Several implementations of the SOCKSv5 standard 
exist to date \cite{rfc1928}, which allow utilization of a remotely hosted
standard-conforming SOCKSv5 proxy server (typically from
a desktop computer through an organizational server). Mobile
implementations, however, are less frequent. One example
of an implementation for the Android operating system is
the anonymity generating Orbot application (see http://www.
torproject.us for more details), which routes traffic into the
TOR network and contains “proxification” methods for applications
as well (i.e., transparently forcing the usage of the
proxy through, e.g., modifications of the iptables firewall). A basic Android service application was generated that is based on
the jSOCKS proxy server implementation \cite{kouzoubov2011}, which is open source
(entirely written in JAVA) and does not require any
privileges, such as root level system access. As the service is
executed within the Dalvik VM utilized on Android devices,
it incurs a minimal computational overhead when compared to
native applications. This approach, however, is commonplace
to allow broadest application compatibility and encouraged for
developers of the platform.

%Need to italicize as is done in paper
%Need to fix equations
\addcontentsline{toc}{subsection}{Performance Metrics}
\subsection*{Performance Metrics}
In the following, we briefly outline the metrics used to evaluate
the performance of either scheme. Initially, we capture
the reported voltage level v(tl) [V] and the current i(tl) [A]
as reported by the power supply and timestamped at time tl
on the connected desktop computer. We similarly calculate
the instant power consumption as p(tl) = v(tl) · i(tl) [W].
As the reported values are instantaneous snapshots in time
from l = 0 at t(l) = 0 (denoting the first measurement) to
l = L, which happened at t(l) = T (whereby T denotes
the last measurement), we calculate the time passed between
consecutive measurement instances as t(l) = tl − tl−1. To
determine the energy that was used in the l-th measurement
period, we calculate e(l) = t(l) · w(tl) [J]. We denote the
energy that was used in a measurement period up to t(l) as
\newline
\newline
Need to add in equations.
\newline

\begin{figure}
\centering
\includegraphics[scale=1,keepaspectratio]{proxy_setup}
\caption{Overview of the measurement setup}
\label{fig:proxy_setup}
\end{figure}

\addcontentsline{toc}{section}{Performance Evaluation For Web Requests}
\subsection*{Performance Evaluation For Web Request}
To perform a representative evaluation of frequent HTTP
web requests, web requests for Google’s home page were utilized.
The goal of this particular measurement scenario is to evaluate
the performance impact of frequent requests through the local
proxy server, which has to perform the additional connection
tasks each time a request is made. A direct
measurement application for Android was developed, which will request http://www.google.com without further resolving any
HTTP objects within. Individual requests are followed by a
sleep period of 2 seconds for both, direct requests and requests
through the mobile SOCKSv5 proxy service. As requests are
made without utilizing a browser, no caching is involved
client-side.

\addcontentsline{toc}{subsection}{Fixed Network}
\subsection*{Fixed Network}
In the fixed network scenario, the Pandaboard is connected
through wired Ethernet to the campus network while performing
the web requests. The requests typically coincide
with high power consumption levels, as illustrated in Figure \ref{fig:proxy_energy_cons_socks}
for an exemplary 100 web requests with both configurations.
We observe that both approaches exhibit an initial “spike”
behavior and an otherwise low level (with some general noise
due to overall device activities). There is no immediately
visible trend for the momentary power consumptions, as in
both approaches, there are several bursty periods of slightly
elevated consumptions on top of the actual web requests.
Next, we evaluate the total (compounded) energy consumption
that is observed when performing these requests for a
certain period of time. We illustrate 100 subsequent requests
directly and through the mobile proxy server in Figure \ref{fig:proxy_comp_energy_cons}.
Initially, it is observed that despite the short-term fluctuations,
there is a linear increase in the energy consumed
while using either approach. More significantly, there is no
immediately noticeable significant difference between the
approaches, which is indicated by the almost indistinguishable
values in the plot. Lastly, comparing the overhead between
the approaches numerically in Table \ref{tab:requests_summary}, where the
300 highest levels of energy consumption measured for periods
of placing 300 requests were analyzed. It's also notable that the direct approach
results in a higher average level of energy consumption (albeit
with a larger variability), whereas the proxy-based approach
yields a lower average and variability of energy usage values
determined. Overall, this results in an overhead of o =
−0.0563, which presents an initially counter-intuitive result.
(Differently worded, by utilizing the mobile proxy server
consuming additional CPU cycles, potential energy savings
of 5.6\% could be realized without requiring any additional
modifications.) Taking into account that partially significant
variability exists due to some outliers in the total duration
(as some measurement points can exhibit significant delays or
coincide with other unrelated system activities), both approaches are very similar.

\begin{figure}
\centering
\includegraphics[scale=1,keepaspectratio]{proxy_energy_cons_socks}
\caption{Energy consumption while performing 100 web requests directly or through a mobile SOCKSv5 service, smoothed over time.}
\label{fig:proxy_energy_cons_socks}
\end{figure}

\begin{figure}
\centering
\includegraphics[scale=1,keepaspectratio]{proxy_comp_energy_cons}
\caption{Compounded energy consumption for performing 100 web requests directly or through a mobile SOCKSv5 service.}
\label{fig:proxy_comp_energy_cons}
\end{figure}

\begin{table}[h]
\begin{tabular}{|c|c|c|c|c|}
\hline
 Interface & Approach  & Average[J]  & Standard Deviation [J] & Confidence Interval (99 \%)  \\ \hline
 LAN & Direct & 0.0912 & 0.0139 & 0.0021 \\ \hline
 LAN & Proxy & 0.0860 & 0.0062 & 0.0009 \\ \hline
 WLAN & Direct & 0.0900 & 0.0125 & 0.0019 \\ \hline
 WLAN & Proxy & 0.0902 & 0.0089 & 0.0013 \\ \hline
\end{tabular}
\caption{Summary values for 300 direct and proxy-routed web requests over traditional ethernet and wireless LAN networks.}
\label{tab:requests_summary}
\end{table}

\addcontentsline{toc}{subsection}{Wireless Network}
\subsection*{Wireless Network}
Shifting the evaluation to HTTP requests made over the
wireless network interface, we present our results in Table \ref{tab:requests_summary}.
(We note that a graphical evaluation would yield results similar
to those presented in Figures \ref{fig:proxy_energy_cons_socks} and \ref{fig:proxy_comp_energy_cons}.) We initially note that
both approaches are very close with respect to their measured
average energy consumption, resulting standard deviations,
and narrow confidence intervals. \newline
Comparing these results with those obtained for the Ethernet
scenario, which outlines the base case without active wireless
communications overheads, we do not note a significant difference
in the average energy usage for the individual web
requests.

\addcontentsline{toc}{section}{Impact of Web Request Variations}
\subsection*{Impact of Web Request Variations}
Motivated by the closeness of requests, the next step is to more
closely evaluate the impact of the web request size over a
wireless LAN on the overall power consumption. To limit
the impact that external networks can have (such as different
delays), the direct performance comparison is performed within
the on-campus VM environment illustrated in Figure \ref{fig:proxy_setup}. A
dedicated virtual machine uses the Apache2 web server and
hosts a Python script that generates a requested number of
bytes, additionally eliminating potential caching impacts. 100 repeated measurements are performed for each different web request size and significant outliers are deleted.

\addcontentsline{toc}{subsection}{Power Consumption}
\subsection*{Power Consumption}

LEFT OFF FROM PAGE 44 IN PROXY PAPER
\chapter{Web Cache Object Forwarding From Desktop to Mobile for Energy Consumption Optimizations}
\addcontentsline{toc}{section}{Introduction}
\section*{Introduction}
With the beginning of the 21st century, networking support for wirelessly connected mobile user devices has fueled a continuous increase in the demand for mobile data.
Web requests now originate in a majority from wirelessly connected user devices -- a trend that Cisco, Inc. predicts to continue in the foreseeable future~\cite{VNI14}.
Simultaneously, the overall user behavior and demand for more rich media inclusion into web pages has increased the overall amount of data that is required to be transmitted per page, see, e.g., \cite{IhPa11,BuMaSe13}.
Caching on the client side has been effectively used in the past and was, together with increased numbers of parallel object downloads, able to decrease wait times for desktop clients as reported in \cite{IhPa11}.
A first view of mobile web page characteristics (which were found to exhibit lower complexity than regular desktop browser versions) and non-landing pages (which were found to be less complex than landing pages) was evaluated by the authors of~\cite{BuMaSe13}.
Moreover, a significant body of research has emerged that focuses on content delivery optimizations to mobile devices.

Typically, these optimization approaches are targeting on-device optimizations or off--device cloud--based improvements.
For mobile applications, for example, significant energy savings were found to be attainable when grouping application requests so as to avoid prolonged cellular network interface activity, see, e.g., \cite{BaBaVe09,QiWaGaHuGe12}.
Other approaches optimize the delivery to mobile devices through proxies and cloud-based data anticipation and traffic shaping, see, e.g., \cite{XiHuSaYl11}.

For web data, typically caching is used to limit the amount of data that has to be transfered to requesting clients.
In prior works focusing on mobile web page delivery optimizations, such as \cite{SaIs02}, energy optimization has been a key element, due to the limiting restrictions for mobile clients.
One particular area for optimization is the pre--fetching of web page objects combined with caching (which allows to download data before usage), such as presented in~\cite{ShKuDaWa05} with about 10 \% energy savings.
More recently, these approaches were combined with user connectivity predictions, see, e.g., \cite{ThChWo13}.

\begin{figure}
\centering
\includegraphics[scale=1,keepaspectratio]{cache_forwarding_setup}
	\caption{Screenshot of rendered web page from WebPageSpeedTest.org. As illustrated, the main textual and pictorial content items are flanked by background and interactive advertisements.}
\label{fig:cache_forwarding_setup}
\end{figure}

We propose to utilize a basic cache forwarding method that can be utilized by users to synchronize from, e.g., a desktop computer, to their mobile device, e.g., a smartphone.
We illustrate this approach in Figure~\ref{fig:cache_forwarding_setup}, which contains the desktop and a mobile device. 
As most devices are charged over night, or are at least stationary within local area network ranges, we propose to utilize this general idle period to allow direct forwarding of cached objects.
We presume that browser clients will have a shared information source in the cloud that allows the identification of visited web pages, an approach most browser clients take to date.
In turn, clients are enabled to identify web page objects that are identical between the different device display modalities, i.e., identical web objects delivered when requesting a desktop/mobile versions of a web page.
The thus transmitted objects, in turn, reside locally within the mobile device cache and do not require an energy--expensive mobile download through cellular interfaces. 
We note that the reverse of this approach is possible as well.
To support our approach, we gather data through the publicly accessible \url{WebpageSpeedTest.org} website as well as the \url{httparchive.org} archive of a large dataset of performance evaluations for popular web pages; we refer the interested reader to~\cite{Me13} for a more detailed discussion.


\chapter{CONCLUSION}

	
\subsection*{Future Work}
\addcontentsline{toc}{subsection}{Future Work}



\end{Spacing}
\bibliographystyle{unsrt}
\bibliography{main}
\end{document}
